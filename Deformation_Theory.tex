\documentclass{article}
\usepackage[utf8]{inputenc}
\usepackage{amsmath}
\usepackage{amsfonts}
\usepackage{amssymb}
\usepackage{tikz}
\usepackage{fullpage}
\usepackage{tikz-cd}
\usepackage{spectralsequences}
\usepackage{adjustbox}
\usepackage{xfrac}
\usepackage{tcolorbox}
\usepackage{xcolor}
\usepackage{hyperref}
\usepackage{graphicx}
\graphicspath{ {D:/Chrome Downloads./} }
\usepackage{amsthm}
\newcommand{\R}{\mathbb{R}}
\renewcommand{\P}{\mathbb{P}}
\newcommand{\C}{\mathbb{C}}
\newcommand{\A}{\mathbb{A}}
\newcommand{\Spec}{\textrm{Spec}}
\newcommand{\Proj}{\textrm{Proj}}
\newcommand{\Hom}{\textrm{Hom}}
\usetikzlibrary{calc}
\theoremstyle{definition}
\newtheorem{theorem}{Theorem}[section]
\theoremstyle{definition}
\newtheorem{definition}{Definition}[theorem]
\theoremstyle{definition}
\newtheorem{remark}{Remark}[theorem]
\theoremstyle{definition}
\newtheorem{proposition}{Proposition}[theorem]
\theoremstyle{definition}
\newtheorem{lemma}[theorem]{Lemma}
\theoremstyle{definition}
\newtheorem{corollary}{Corollary}[theorem]
\theoremstyle{definition}
\newtheorem{example}{Example}[theorem]
\tikzset{curve/.style={settings={#1},to path={(\tikztostart)
    .. controls ($(\tikztostart)!\pv{pos}!(\tikztotarget)!\pv{height}!270:(\tikztotarget)$)
    and ($(\tikztostart)!1-\pv{pos}!(\tikztotarget)!\pv{height}!270:(\tikztotarget)$)
    .. (\tikztotarget)\tikztonodes}},
    settings/.code={\tikzset{quiver/.cd,#1}
        \def\pv##1{\pgfkeysvalueof{/tikz/quiver/##1}}},
    quiver/.cd,pos/.initial=0.35,height/.initial=0}
\title{Deformation Theory}
\author{David Zhu}

\begin{document}
\maketitle


\tableofcontents

\newpage



\section{Introduction}
In the section, we will introduce the definitions and intuition for first order deformations. 

\subsection{Intuition}
\textbf{Embedded Deformation}: Suppose we have a smooth submanifold $X$ embedded an ambient complex manifold $Y$. The embedding is equipped with a normal bundle $N_YX$. By the tubular neighborhood theorem, we have a embedding of its total space: 

\[\begin{tikzcd}
	X && Y \\
	& {N_YX}
	\arrow[hook, from=1-1, to=1-3]
	\arrow["{\textrm{0-section}}"', from=1-1, to=2-2]
	\arrow[dotted, hook, from=2-2, to=1-3]
\end{tikzcd}\]

A smooth deformation of $Y$ inside $X$ is then a smooth section of $N_YX$: at each point $x\in X$, the section give you the normal direction along which to ``infinitesimally" deform $X$ inside $Y$. This definition offers some differential topologically intuition, even though we no longer have an analog of the tubular neighborhood theorem in the holomorphic/algebraic setting.

More generally, we are given the data of 
\begin{enumerate}
    \item A morphism of objects $f: X\to Y$ in some category (e.g. schemes, complex manifolds, etc).
    \item An ``infinitesimal thickening" of $X$ and $Y$, which are prescribed injective morphisms $X\to X'$ and $Y\to Y'$.
\end{enumerate}

A deformation of $f$ is then a lift of the morphism $X\to Y$ to a morphism $X'\to Y'$
\[\begin{tikzcd}
	X & Y \\
	{X'} & {Y'}
	\arrow[from=1-1, to=1-2]
	\arrow[from=1-1, to=2-1]
	\arrow[from=1-2, to=2-2]
	\arrow["{?}"', dotted, from=2-1, to=2-2]
\end{tikzcd}\]

\textbf{Deformation of Complex Structure}: Viewing a complex manifold $X$ as a real manifold with an integrable almost complex structure $J$, a deformation of the complex structure is a family of almost complex structures $J_t$ parametrized by $t\in (-\epsilon, \epsilon)$ such that $J_0 = J$ and each $J_t$ is integrable. Infinitesimally, we can think of this as a perturbation of the almost complex structure $J$ by a small parameter $t$ in the space of endomorphisms of the tangent bundle $TX$ that satisfy $J_t^2 = -\mathrm{Id}$ and the integrability condition (vanishing of the Nijenhuis tensor).


The theory of Kodaira-Spencer describes how such deformations can be understood in terms of certain cohomology groups associated with the manifold. More precisely, first order deformations of the complex structure on $X$ are classified by the cohomology group $H^1(X, T_X)$, where $T_X$ is the holomorphic tangent bundle of $X$.





\textbf{Deformation and Moduli Problem}
In relation to deformation of complex structure, suppose we have concretely constructed the moduli space $\mathcal{M}_g$ parameterizing smooth projective curves of genus $g$. A point $[C]\in \mathcal{M}_g$ corresponds to an isomorphism class of a smooth projective curve $C$ of genus $g$. A deformation of the curve $C$ can be thought of as a small perturbation of the complex structure on $C$, leading to a family of curves $C_t$ parameterized by a small parameter $t$. Infinitesimally, this corresponds to moving along a tangent vector at the point $[C]$ in the moduli space $\mathcal{M}_g$. The tangent space at $[C]$ can be identified with the first cohomology group $H^1(C, T_C)$, where $T_C$ is the holomorphic tangent bundle of the curve $C$.

Suppose we have a fine moduli space $\mathcal{M}$ parameterizing certain objects, meaning any family of such objects over a base $B$ is given by a morphism 
\[B\to \mathcal{M}\]
Then an infinitesimal deformation of an object $X$ corresponding to a point $[X]\in \mathcal{M}$ is given by a family over the dual numbers $\mathrm{Spec}(\mathbb{C}[\epsilon]/(\epsilon^2))$ (we can think of the dual numbers as a point equipped with a tangent vector), such that the fiber over $0$ is isomorphic to $X$. Then, the classifying map 
\[T_0B\to T_{[X]}\mathcal{M}\]
precisely specifies a tangent vector at $[X]$.

\newpage
\section{First Order Deformation}
\begin{definition}
	The \textbf{dual numbers} over a field $k$ is the ring 
	\[D:= k[t]/(t^2)\]
\end{definition} 

Note that $\Spec(D)$ is the one-point space consisting of the prime ideal $(t)$. Its Zariski tangent space is given by 
\[T_{(t)}D:= \Hom_k ((t)/(t^2),k) \]
which is a one-dimensional $k$-vector space. In contrast, $\Spec(k)$ is also a one-point space, but its Zariski tangent space is zero-dimensional. Thus, we can think of $\Spec(D)$ as a point equipped with a tangent vector. The dual numbers will be our base space for first order deformations.


\begin{definition}
	\label{deformation_def}
	Let $Y$ be a scheme over $k$, and $X$ be a closed subscheme. A \textbf{first order embedded deformation} of $X$ in $Y$ is a closed subscheme $X'\subset Y\times_k \Spec (D)$ such that 
	\begin{enumerate}
		\item $X'$ is flat over $D$
		\item The fiber over the point is $X$, i.e 
		\[X'\times_{\Spec(D)}\Spec(k)= X\]
	\end{enumerate}
\end{definition}

We would like to classify all such deformations. It is useful to characterize flatness over the dual numbers first: 

\begin{lemma}
	\label{flatness_criterion}
	A module $M$ over a commutative ring $R$ is flat iff for every ideal $I\subset R,$ the module homomorphism 
	\[I\otimes_R M\to M\]
	is injective.
\end{lemma}
\begin{proof}
	Follows directly from the fact that flatness is equivalent to the vanishing of $\mathrm{Tor}_1^R(R/I, M)$ for all ideals $I\subset R$.
\end{proof}







\subsection{Embedded Deformation of Schemes}
We first deal with the affine case. Suppose $X=\Spec(B)$ is affine, and $Y$ is a closed subscheme defined by an ideal $I\subset B$. We see that first-order embedded deformations of $Y$ in $X$ correspond to ideals $J\subset B':= B[t]/(t^2)$ such that 
\begin{enumerate}
	\item $B'/J$ is flat over $D$
	\item The image of $J$ under the quotient map
	\[B[t]/(t^2)\to B\]
	is $I$.
\end{enumerate}
\begin{proposition}
\label{affine_deformation}
	There is a bijection between the set of first-order embedded deformations of $Y$ in $X$ and $\mathrm{Hom}_B (I, B/I)$.
\end{proposition}
\begin{proof}


	Suppose we are given an ideal $J\subset B'$ such that the $J \textrm{ mod } t =I$. By Lemma \ref{flatness_criterion}, flatness of $B'/J$ over $D$ is equivalent to the exactness of
	\[ 0\to B'/J\otimes_D (t)\to B'/J\to B'/J\otimes_D k\to 0\]
	Viewing $(t)$ as a $k$-module, we have 
	\[B/I\otimes_k (t)\cong (B'/J\otimes_D k)\otimes_k (t)\cong  B'/J\otimes_D (t)\]
	So we may rewrite the exact sequence as
	\[0\to B/I\xrightarrow{\cdot t} B'/J\to B/I\to 0\]
	Since $B':=B[t]/(t^2)$ splits as $B\oplus tB$ as $B$-modules, for every $f\in I$, there exist lifts of the form $f+tg\in J$, where $f,g\in B$. By exactness of the sequence above, different lifts differ by an element in $tI$, so we may then define a function
	\[\varphi: I\to B/I\]
	by $\varphi(f)=\overline{g}$. It is easy to check this is well-defined and a $B$-module homomorphism. Conversely, given a $B$-module homomorphism $\phi: I\to B/I$, we may define an ideal:
	\[J:= \{f+t g\mid f\in I, g\in B, g\textrm{ mod } I = \phi(f)\}\]
	Flatness of $B'/J$  over $D$ can be checked by by the same exact sequence as above: an element $tb$ is in the ideal $J$ iff $b\in I$. The two constructions are inverse to each other, and we have the desired bijection.

\end{proof}

\begin{definition}
	The first order embedded deformations of $Y$ in $X$ is \textbf{trivial} if they correspond to the zero homomorphism in $\mathrm{Hom}_B (I, B/I)$, which corresponds to the ideal $J=I\oplus tI\subset B[t]/(t^2)$.
\end{definition}


\begin{example}
We are given curve $C$ carved out by $f(x,y)$, embedded in $\mathbb{A}^2$ via the canonical map 
	 \[k[x,y]\to k[x,y]/(f)\]
By our above analysis, first order embedded deformations of $C$ in $\mathbb{A}^2$ correspond to elements of 
\[\mathrm{Hom}_{k[x,y]/(f)} ((f), k[x,y]/(f))\cong k[x,y]/(f)\]
In particular, each $g\in k[x,y]/(f)$ gives rise to a first order deformation defined by the principal ideal $J=(f+tg)$. We can think of this as perturbing $f$ by $g$ infinitesimally.

\end{example}

Recall that for a closed embedding of schemes $X\to Y$ defined by an ideal sheaf $\mathcal{I}\subset \mathcal{O}_Y$, the normal sheaf $\mathcal{N}_{X/Y}$ is defined as the dual of the conormal sheaf $\mathcal{I}/\mathcal{I}^2$. Moreover, 
\[\Hom_Y(\mathcal{I}, \mathcal{O}_X)\cong \Hom_X(\mathcal{I}/\mathcal{I}^2, \mathcal{O}_X)\] Since flatness is a local property, and our construction above is natural and compatible with localization, we may globalize Proposition \ref{affine_deformation} to obtain:
\begin{theorem}
	Let $Y$ be a scheme over a field $k$, and let $X$ be a closed subscheme of $Y$. Then the deformations of $X$ over $D$ in $Y$ are in natural one-to-one correspondence with elements of $H^0(X,N_{X/Y})$, with the zero element corresponding to the trivial deformation.
\end{theorem}

\subsection{Deformation of Coherent Sheaves}
Let $X$ be a scheme over a field $k$, and $\mathcal{F}$ be a coherent sheaf on $X$. A first order deformation of $\mathcal{F}$ is a coherent sheaf $\mathcal{F}'$ on $X':=X\times_k \Spec(D)$ that is 
\begin{enumerate}
	\item flat over $D$;
	\item equipped with a homomorphism  $\mathcal{F'}\to \mathcal{F}$ such that the induced map $\mathcal{F}'\otimes_D k\to \mathcal{F}$ is an isomorphism.
\end{enumerate}
Two such deformations $\mathcal{F}',\mathcal{F}''$ are considered isomorphic if there exists an isomorphism $\mathcal{F}'\to \mathcal{F}''$ compatible with their morphisms to $\mathcal{F}$. 

\begin{theorem}
	Let $X$ be a scheme over a field $k$, and $\mathcal{F}$ be a coherent sheaf on $X$. Then the isomorphism classes of first order deformations of $\mathcal{F}$ are in bijection with elements of $\mathrm{Ext}^1_X(\mathcal{F}, \mathcal{F})$, with the zero element corresponding to the trivial deformation.
\end{theorem}
\begin{proof}
	By Lemma \ref{flatness_criterion}, a first order deformation $\mathcal{F}'$ of $\mathcal{F}$ fits into a short exact sequence of $D$-modules
	\[0\to \mathcal{F}\xrightarrow{t} \mathcal{F}'\to \mathcal{F}\to 0\]
	Note that tensoring the split exact sequence 
\[0\to k\to D\to k\to 0\]
with $\mathcal{O}_X$, we get canonical splitting $\mathcal{O}_X\to \mathcal{O}_X'$, so we may view $\mathcal{F}'$ as an $\mathcal{O}_X$-module. Thus, the above short exact sequence of $\mathcal{O}_X$-modules is an element of $\mathrm{Ext}^1_X(\mathcal{F}, \mathcal{F})$. Conversely, given an extension, we may define a $\mathcal{O}_{X'}$-module structure on it by letting $t$ act via first projecting along $\mathcal{F}'\to \mathcal{F}$, and then apply $\mathcal{F}\xrightarrow{t} \mathcal{F}'$. The two constructions are inverse to each other, giving the desired bijection.

\end{proof} 

We may apply the theorem to the special case of vector bundles, i.e locally free sheaves. 

\begin{corollary}
	\label{def_vector_bundle}
	The first order deformation of a vector bundle $\mathcal{E}$ on $X$ are classified by elements of $H^1(X, \mathcal{E}nd(\mathcal{E}))$, with the trivial deformation corresponding to the zero element.
\end{corollary}
\begin{proof}
	Note that $H^1(X, \mathcal{E}nd(\mathcal{E}))\cong \textrm{Ext}^1(\mathcal{O}_X,  \mathcal{E}nd(\mathcal{E}))$, since $\Gamma(X, -)= \Hom(\mathcal{O}_X, -)$ so their derived functors are isomorphic. By \cite{Hartshorne}, Prop 6.7, we have the adjunction 
	\[\textrm{Ext}^1(\mathcal{O}_X,  \mathcal{E}nd(\mathcal{E}))\cong \textrm{Ext}^1(\mathcal{E},  \mathcal{E})\].
\end{proof}

\begin{example}
	We know that the moduli functor of vector bundles on a scheme $X$ is not representable, i.e there is no fine moduli space of vector bundles over $X$. One may see this via the existence of non-trivial automorphisms of trivial vector bundle, as we can glue trivial families together and get a non-trivial family with isomorphic fibers. Yet, we can still study the local structure of the moduli problem via deformation theory: Let $X=\P ^1$, and $\mathcal{E}= \mathcal{O}(1)\oplus \mathcal{O}(-1)$. It is easy to compute that $\mathcal{E}nd(\mathcal{E})\cong \mathcal{O}(2)\oplus \mathcal{O}(-2)\oplus 2$. By Corollary \ref{def_vector_bundle}, we see the deformations of $\mathcal{E}$ are in bijection with elements of 
	\[ H^1(\P^1, \mathcal{E}nd(\mathcal{E}))\cong k\]
	coming from the $\mathcal{O}(-2)$ summand. The non-trivial infinitesimal deformation can be seen in the family given by extensions 
	\[0\to \mathcal{O}(-1)\to \mathcal{E}_t\to \mathcal{O}(1)\to 0\]
	parametrized by $t\in \textrm{Ext}^1(\mathcal{O}(-1), \mathcal{O}(1))\cong k$. By Grothendieck's theorem, any vector bundle on $\P^1$ splits as a direct sum of line bundles, and the additivity of Chern class forces $\mathcal{E}_t\cong \mathcal{O}(-d)\oplus \mathcal{O}(d)$. Moreover, $\mathcal{O}(-1)$ admits a non-zero morphism to $\mathcal{O}(k)$ iff $k\geq -1$, so the only possibilities are $d=0,1$. The case where $d=1$ corresponds to the trivial extension $\mathcal{E}_0=\mathcal{E}$, and the case $d=0$ gives us the non-trivial extension when $t\neq 0$. 


\end{example}

\section{Cohomology of Commutative Rings}
We are now interested in studying more general deformations of schemes, not necessarily embedded in an ambient space. 
\begin{definition}
	Let $X$ be a $k$-scheme, and $A$ an Artin ring over $k$. A \textbf{deformation} of $X$ over $A$ is a scheme $X'$ flat over $A$, together with a closed immersion $i: X\to X'$ such that the induced map 
	\[i\times_A k: X\to X'\times_A k\]
is an isomorphism. Tow such deformations are \textbf{equivalent} if there is an isomorphism $\phi: X'\to X''$ compatible with the closed immersions of $X$.



\end{definition}

We then recall some facts on K\"ahler differentials. Given a morphism of commutative rings $A\to B$, the relative K\"ahler differentials $\Omega_{B/A}$ is defined to be the $B$-module generated by symbols $db$ for $b\in B$, subject to the relations
\begin{itemize}
	\item $da=0$,
	\item $d(a_1+a_2)=da_1+da_2$,
	\item $d(ab)=a (db)+b(da)$
\end{itemize}
The construction is compatible with localization, so we may globalize to general schemes. Given a morphism of schemes 
\[X\xrightarrow{f} Y\xrightarrow{g} Z\]
there is a natural exact sequence of sheaves on $X$:
\[f^*\Omega_{Y/Z}\to \Omega_{X/Z}\to \Omega_{X/Y}\to 0\]
and we wish to extend such exact sequences to the left. Historically, Lichtenbaum and Schleissinger directly constructed the first $3$ functors $T^0,T^1,T^2$ that continued the long exact sequence; later Andr\'e and Quillen generalized the construction to all $T^i$ using simplicial methods. In particular, Quillen was able to develop the theory in the general framework of model categories in which derived functors can be defined, even when the underlying category is no longer abelian.

\subsection{ \texorpdfstring{The $T^i$ Functors}{}  }
The construction starts with making the following choices: given a ring homomorphism $A\to B$, choose a polynomial ring $R=A[x_1,x_2,...]$ large enough to admit an $A$-algebra surjection $R\to B$. Let $I$ be the ideal defining $B$, so we have a short exact sequence
\[0\to I\to R\to B\to 0\]
Then, choose a free $R$-module $F$ and a $R$-module surjection $F\to I$, and let $Q$ be the kernel, so we have a short exact sequence
\[0\to Q\to F\xrightarrow{j} I\to 0\]
Now define the $F_0$ as the submodule of $F$ generated by elements of the form $j(a)b-j(b)a$ for all $a,b\in F$. Since $j$ is a $R$-module homomorphism, we see $j(F_0)=0$ so that $F_0\subseteq Q$. 
\begin{definition}
	The \textbf{cotangent complex} of $B$-modules is defined to be the complex
	\[L_2\xrightarrow{d_2}L_1\xrightarrow{d_1}L_0\]
where $L_2=Q/F_0$, $L_1=F\otimes_R B\cong F/IF$ and $L_0=\Omega_{R/A}\otimes_R B$. The differential $d_2$ is induced by the inclusion $Q\to F$. The differential $d_1$ is defined by the composition 
\[F/IF\xrightarrow{j}I/I^2\xrightarrow{d}\Omega_{R/A}\to \Omega_{R/A}\otimes_R B \]
\end{definition}

\begin{definition}
	For any $B$-module $M$, define 
	\[T^i(B/A,M):= H^i(\Hom_B(L_{\bullet}, M))\]
	
\end{definition}






























\newpage
\bibliographystyle{plain}
\bibliography{citation}
\end{document}