\documentclass{article}
\usepackage[utf8]{inputenc}
\usepackage{amsmath}
\usepackage{amsfonts}
\usepackage{amssymb}
\usepackage{tikz}
\usepackage{fullpage}
\usepackage{tikz-cd}
\usepackage{spectralsequences}
\usepackage{adjustbox}
\usepackage{xfrac}
\usepackage{tcolorbox}
\usepackage{xcolor}
\usepackage{hyperref}
\usepackage{graphicx}
\graphicspath{ {D:/Chrome Downloads./} }
\usepackage[parfill]{parskip}
\usepackage{amsthm}
\usetikzlibrary{calc}
\theoremstyle{definition}
\newtheorem{theorem}{Theorem}[section]
\theoremstyle{definition}
\newtheorem{definition}{Definition}[theorem]
\theoremstyle{definition}
\newtheorem{remark}{Remark}[theorem]
\theoremstyle{definition}
\newtheorem{proposition}{Proposition}[theorem]
\theoremstyle{definition}
\newtheorem{lemma}[theorem]{Lemma}
\theoremstyle{definition}
\newtheorem{corollary}{Corollary}[theorem]
\theoremstyle{definition}
\newtheorem{example}{Example}[theorem]
\tikzset{curve/.style={settings={#1},to path={(\tikztostart)
    .. controls ($(\tikztostart)!\pv{pos}!(\tikztotarget)!\pv{height}!270:(\tikztotarget)$)
    and ($(\tikztostart)!1-\pv{pos}!(\tikztotarget)!\pv{height}!270:(\tikztotarget)$)
    .. (\tikztotarget)\tikztonodes}},
    settings/.code={\tikzset{quiver/.cd,#1}
        \def\pv##1{\pgfkeysvalueof{/tikz/quiver/##1}}},
    quiver/.cd,pos/.initial=0.35,height/.initial=0}
\title{Deformation Theory}
\author{David Zhu}

\begin{document}
\maketitle


\tableofcontents

\newpage



\section{Introduction}
In the section, we will introduce the definitions and intuition for first order deformations. 

\subsection{Intuition}
\textbf{Embedded Deformation}: Suppose we have a smooth submanifold $X$ embedded an ambient complex manifold $Y$. The embedding is equipped with a normal bundle $N_YX$. By the tubular neighborhood theorem, we have a embedding of its total space: 

\[\begin{tikzcd}
	X && Y \\
	& {N_YX}
	\arrow[hook, from=1-1, to=1-3]
	\arrow["{\textrm{0-section}}"', from=1-1, to=2-2]
	\arrow[dotted, hook, from=2-2, to=1-3]
\end{tikzcd}\]

A smooth deformation of $Y$ inside $X$ is then a smooth section of $N_YX$: at each point $x\in X$, the section give you the normal direction along which to ``infinitesimally" deform $X$ inside $Y$. This definition offers some differential topologically intuition, even though we no longer have an analog of the tubular neighborhood theorem in the holomorphic/algebraic setting.

More generally, we are given the data of 
\begin{enumerate}
    \item A morphism of objects $f: X\to Y$ in some category (e.g. schemes, complex manifolds, etc).
    \item An ``infinitesimal thickening" of $X$ and $Y$, which are prescribed injective morphisms $X\to X'$ and $Y\to Y'$.
\end{enumerate}

A deformation of $f$ is then a lift of the morphism $X\to Y$ to a morphism $X'\to Y'$
\[\begin{tikzcd}
	X & Y \\
	{X'} & {Y'}
	\arrow[from=1-1, to=1-2]
	\arrow[from=1-1, to=2-1]
	\arrow[from=1-2, to=2-2]
	\arrow["{?}"', dotted, from=2-1, to=2-2]
\end{tikzcd}\]

\textbf{Deformation of Complex Structure}: 





\textbf{Deformation as a Family}
Given an object $X$ in some category, a deformation of $X$ can also be thought of as a family of objects $\mathcal{X}\to S$ over some parameterizing base $S$, such that the fiber over a distinguished point $s_0\in S$ is isomorphic to $X$.























































\newpage
\bibliographystyle{plain}
\bibliography{citation}
\end{document}