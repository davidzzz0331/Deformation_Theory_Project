\documentclass{article}
\usepackage[utf8]{inputenc}
\usepackage{amsmath}
\usepackage{amsfonts}
\usepackage{amssymb}
\usepackage{tikz}
\usepackage{fullpage}
\usepackage{tikz-cd}
\usepackage{spectralsequences}
\usepackage{adjustbox}
\usepackage{xfrac}
\usepackage{tcolorbox}
\usepackage{xcolor}
\usepackage{hyperref}
\usepackage{graphicx}
\graphicspath{ {D:/Chrome Downloads./} }
\usepackage[parfill]{parskip}
\usepackage{amsthm}
\newcommand{\R}{\mathbb{R}}
\newcommand{\C}{\mathbb{C}}
\newcommand{\Spec}{\textrm{Spec}}
\newcommand{\Proj}{\textrm{Proj}}
\newcommand{\Hom}{\textrm{Hom}}
\usetikzlibrary{calc}
\theoremstyle{definition}
\newtheorem{theorem}{Theorem}[section]
\theoremstyle{definition}
\newtheorem{definition}{Definition}[theorem]
\theoremstyle{definition}
\newtheorem{remark}{Remark}[theorem]
\theoremstyle{definition}
\newtheorem{proposition}{Proposition}[theorem]
\theoremstyle{definition}
\newtheorem{lemma}[theorem]{Lemma}
\theoremstyle{definition}
\newtheorem{corollary}{Corollary}[theorem]
\theoremstyle{definition}
\newtheorem{example}{Example}[theorem]
\tikzset{curve/.style={settings={#1},to path={(\tikztostart)
    .. controls ($(\tikztostart)!\pv{pos}!(\tikztotarget)!\pv{height}!270:(\tikztotarget)$)
    and ($(\tikztostart)!1-\pv{pos}!(\tikztotarget)!\pv{height}!270:(\tikztotarget)$)
    .. (\tikztotarget)\tikztonodes}},
    settings/.code={\tikzset{quiver/.cd,#1}
        \def\pv##1{\pgfkeysvalueof{/tikz/quiver/##1}}},
    quiver/.cd,pos/.initial=0.35,height/.initial=0}
\title{Deformation Theory}
\author{David Zhu}

\begin{document}
\maketitle


\tableofcontents

\newpage



\section{Introduction}
In the section, we will introduce the definitions and intuition for first order deformations. 

\subsection{Intuition}
\textbf{Embedded Deformation}: Suppose we have a smooth submanifold $X$ embedded an ambient complex manifold $Y$. The embedding is equipped with a normal bundle $N_YX$. By the tubular neighborhood theorem, we have a embedding of its total space: 

\[\begin{tikzcd}
	X && Y \\
	& {N_YX}
	\arrow[hook, from=1-1, to=1-3]
	\arrow["{\textrm{0-section}}"', from=1-1, to=2-2]
	\arrow[dotted, hook, from=2-2, to=1-3]
\end{tikzcd}\]

A smooth deformation of $Y$ inside $X$ is then a smooth section of $N_YX$: at each point $x\in X$, the section give you the normal direction along which to ``infinitesimally" deform $X$ inside $Y$. This definition offers some differential topologically intuition, even though we no longer have an analog of the tubular neighborhood theorem in the holomorphic/algebraic setting.

More generally, we are given the data of 
\begin{enumerate}
    \item A morphism of objects $f: X\to Y$ in some category (e.g. schemes, complex manifolds, etc).
    \item An ``infinitesimal thickening" of $X$ and $Y$, which are prescribed injective morphisms $X\to X'$ and $Y\to Y'$.
\end{enumerate}

A deformation of $f$ is then a lift of the morphism $X\to Y$ to a morphism $X'\to Y'$
\[\begin{tikzcd}
	X & Y \\
	{X'} & {Y'}
	\arrow[from=1-1, to=1-2]
	\arrow[from=1-1, to=2-1]
	\arrow[from=1-2, to=2-2]
	\arrow["{?}"', dotted, from=2-1, to=2-2]
\end{tikzcd}\]

\textbf{Deformation of Complex Structure}: Viewing a complex manifold $X$ as a real manifold with an integrable almost complex structure $J$, a deformation of the complex structure is a family of almost complex structures $J_t$ parametrized by $t\in (-\epsilon, \epsilon)$ such that $J_0 = J$ and each $J_t$ is integrable. Infinitesimally, we can think of this as a perturbation of the almost complex structure $J$ by a small parameter $t$ in the space of endomorphisms of the tangent bundle $TX$ that satisfy $J_t^2 = -\mathrm{Id}$ and the integrability condition (vanishing of the Nijenhuis tensor).


The theory of Kodaira-Spencer describes how such deformations can be understood in terms of certain cohomology groups associated with the manifold. More precisely, first order deformations of the complex structure on $X$ are classified by the cohomology group $H^1(X, T_X)$, where $T_X$ is the holomorphic tangent bundle of $X$.





\textbf{Deformation and Moduli Problem}
In relation to deformation of complex structure, suppose we have concretely constructed the moduli space $\mathcal{M}_g$ parameterizing smooth projective curves of genus $g$. A point $[C]\in \mathcal{M}_g$ corresponds to an isomorphism class of a smooth projective curve $C$ of genus $g$. A deformation of the curve $C$ can be thought of as a small perturbation of the complex structure on $C$, leading to a family of curves $C_t$ parameterized by a small parameter $t$. Infinitesimally, this corresponds to moving along a tangent vector at the point $[C]$ in the moduli space $\mathcal{M}_g$. The tangent space at $[C]$ can be identified with the first cohomology group $H^1(C, T_C)$, where $T_C$ is the holomorphic tangent bundle of the curve $C$.

Suppose we have a fine moduli space $\mathcal{M}$ parameterizing certain objects, meaning any family of such objects over a base $B$ is given by a morphism 
\[B\to \mathcal{M}\]
Then an infinitesimal deformation of an object $X$ corresponding to a point $[X]\in \mathcal{M}$ is given by a family over the dual numbers $\mathrm{Spec}(\mathbb{C}[\epsilon]/(\epsilon^2))$ (we can think of the dual numbers as a point equipped with a tangent vector), such that the fiber over $0$ is isomorphic to $X$. Then, the classifying map 
\[T_0B\to T_{[X]}\mathcal{M}\]
precisely specifies a tangent vector at $[X]$.

\newpage
\section{First Order Deformation}
\begin{definition}
	The \textbf{dual numbers} over a field $k$ is the ring 
	\[D:= k[t]/(t^2)\]
\end{definition}
Note that $\Spec(D)$ is the one-point space consisting of the prime ideal $(t)$. Its Zariski tangent space is given by 
\[T_{(t)}D:= \Hom_k ((t)/(t^2),k) \]
which is a one-dimensional $k$-vector space. In contrast, $\Spec(k)$ is also a one-point space, but its Zariski tangent space is zero-dimensional. Thus, we can think of $\Spec(D)$ as a point equipped with a tangent vector. The dual numbers will be our base space for first order deformations.


\begin{definition}
	Let $Y$ be a scheme over $k$, and $X$ be a closed subscheme. A \textbf{first order embedded deformation} of $X$ in $Y$ is a closed subscheme $X'\subset Y\times_k \Spec (D)$ such that 
	\begin{enumerate}
		\item $X'$ is flat over $D$
		\item The fiber over the point is $X$, i.e 
		\[X'\times_{\Spec(D)}\Spec(k)= X\]
	\end{enumerate}
\end{definition}
We would like to classify all such deformations. 

\subsection{Computations For Affine Curves}
































































\newpage
\bibliographystyle{plain}
\bibliography{citation}
\end{document}